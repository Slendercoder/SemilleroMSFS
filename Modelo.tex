\documentclass{article}
\usepackage[utf8]{inputenc}
\usepackage[spanish]{babel}
\usepackage{amssymb, amsmath,bm,amsthm,enumitem}

\newtheorem{teor}{Teorema}
\newtheorem{lema}{Lema}
\newtheorem{obs}{Observación}

\begin{document}

Motivación: 
\begin{itemize}
\item Sistema de control de temperatura de un panal de abejas. Sistema de recolección de alimento de un hormiguero. Interacción entre las acciones de los individuos y un ``estado macro'' (temperatura, rastros de feromonas en el entorno, etc.).
\item Desarrollar un sistema formal sencillo para  estudiar la interacción elemental entre el estado micro (p.ej., la colección de los estados de las abejas en un instante determinado), y un estado macro (p.ej., la temperatura del panal) (Elementary Micro Macro Interaction, EMMI). 
\item Comparación con los autómatas celulares.
\end{itemize}

\

Definición de un EMMI:
\begin{itemize}
\item Sea $\mathcal{I}=\{1,\ldots,I\}$ un conjunto de agentes. Para cada $i\,{\in}\,\mathcal{I}$ se define:
	\begin{enumerate}[label=(\alph*)]
	\item Umbral $u_i\in[0,1]$.
	\item Estado $x_i[k]\in\{0, 1\}$, para $k\in\mathbb{N}$.
	\item Regla  $x_i[k+1]=\begin{cases} 1, & \mbox{ si }\frac{X[k]}{I}\leq u_i\\ 0, & \mbox{ si } u_i < \frac{X[k]}{I}\end{cases}$
	\end{enumerate}

\item Estado macro: $X[k]=\sum_{i\in\mathcal{I}} x_i[k]$, para $k\in\mathbb{N}$.
\end{itemize}

\begin{itemize}
\item $\bm{u}[k]= (u_1[k],\ldots,u_I[k])$
\item $\bm{x}[k]= (x_1[k],\ldots,x_I[k])$ (estado micro)
\end{itemize}

\begin{obs}
La dinámica del sistema es determinista, es decir, si $\bm{x}[k]=\bm{x}[l]$, entonces $\bm{x}[k{+}1]=\bm{x}[l{+}1]$, para todo $k,l\,{\in}\,\mathbb{N}$.
\end{obs}

\begin{obs}
Si $X[k]=X[l]$, entonces $X[k{+}1]=X[l{+}1]$, para todo $k,l\,{\in}\,\mathbb{N}$.
\end{obs}

\begin{obs}
$X[k{+}1]=|\{i\,{\in}\,\mathcal{I}:\frac{X[k]}{I}\leq u_i\}|$
\end{obs}

\begin{lema}\label{lema1}
No puede haber dos transiciones consecutivas a estados de mayor número de individuos. Es decir, si $X[k]\,{\leq}\,X[k{+}1]$, entonces $X[k{+}2]\,{\leq}\, X[k{+}1]$.
\end{lema}
\begin{proof}
Supongamos que $X[k]\,{\leq}\,X[k{+}1]$. Vamos a demostrar primero que para todo $i\,{\in}\,\mathcal{I}$ se tiene que $x_i[k{+}2]\,{\leq}\, x_i[k{+}1]$. Sea $i$ arbitrario y observe que $x_i[k{+}1]\,{\in}\,\{0, 1\}$. Consideremos cada caso por aparte:
%
\begin{itemize}
\item Supongamos que $x_i[k{+}1]=0$. Luego, por la definición de $x_i[k{+}1]$ (ver (c) arriba) y por la hipótesis se tiene que $u_i\,{<}\,\frac{X[k]}{I}\,{\leq}\,\frac{X[k{+}1]}{I}$. Es decir, $u_i\,{<}\,\frac{X[k{+}1]}{I}$ y, de nuevo por (c) aplicado a $k{+}2$, se tiene que $x_i[k{+}2]\,{=}\,0$. Por lo tanto $x_i[k{+}2]\,{\leq}\,x_i[k{+}1]$.
\item Supongamos que $x_i[k{+}1]\,{=}\,1$. Como $x_i[k{+}2]\,{\in}\,\{0,1\}$, entonces $x_i[k{+}2]\leq x_i[k{+}1]$.
\end{itemize}
%
Como $i$ es arbitrario, entonces $x_i[k{+}2]\,{\leq}\, x_i[k{+}1]$ para todo $i\,{\in}\,\mathcal{I}$. En consecuencia, $\sum_{i\in\mathcal{I}} x_i[k{+}2]\leq \sum_{i\in\mathcal{I}} x_i[k{+}1]$. Por lo tanto, por la definición de $X[k]$ se tiene que $X[k{+}2]\leq X[k{+}1]$.
\end{proof}

\begin{lema}\label{lema2}
No puede haber dos transiciones consecutivas a estados de menor número de individuos. Es decir, si $X[k{+}1]\,{\leq}\,X[k]$, entonces $X[k{+}1]\,{\leq}\, X[k{+}2]$.
\end{lema}

\begin{teor}
No existen ciclos de longitud 3. Es decir, no existe $k\,{\in}\,\mathbb{N}$ tal que $X[k]\,{\neq}\, X[k{+}1]\,{\neq}\, X[k{+}2]$ y $X[k{+}3]\,{=}\,X[k]$. 
\end{teor}
\begin{proof}
Para todo $k\,{\in}\,\mathbb{N}$ vamos a demostrar que si $X[k]\,{\neq}\, X[k{+}1]\,{\neq}\, X[k{+}2]$, entonces $X[k{+}3]\,{\neq}\,X[k]$. Sea $k$ arbitrario y supongamos la hipótesis. Consideremos por aparte los casos $X[k]\,{<}\, X[k{+}1]$ y $X[k]\,{>}\, X[k{+}1]$:
%
\begin{itemize}
\item Supongamos que $X[k]\,{<}\, X[k{+}1]$. Por el lema \ref{lema1} se sigue que $X[k{+}2]\,{\leq}\, X[k{+}1]$. Tenemos dos casos para comparar $X[k]$ y $X[k{+}2]$ (ya sabemos que, por hipótesis, ellos son distintos):
	\begin{itemize}
	\item Caso $X[k{+}2]\,{<}\,X[k]$. Supongamos por absurdo que $X[k{+}3]\,{=}\,X[k]$. Por la observación 2 se sigue que $X[k{+}4]\,{=}\,X[k{+}1]$. Entonces, como arriba supusimos que $X[k]\,{<}\, X[k{+}1]$, se sigue que $X[k{+}2]\,{<}\,X[k{+}3]\,{<}\,X[k{+4}]$. Esto contradice el lema 1. Concluimos que $X[k{+}3]\,{\neq}\,X[k]$.
	\item Caso $X[k]\,{<}\,X[k{+}2]$. Supongamos por absurdo que $X[k{+}3]\,{=}\,X[k]$. Entonces, como arriba vimos que $X[k{+}2]\,{\leq}\, X[k{+}1]$, se sigue que $X[k{+}3]\,{<}\,X[k{+}2]\,{\leq}\,X[k{+1}]$. Esto contradice el lema \ref{lema2}. Concluimos que $X[k{+}3]\,{\neq}\,X[k]$.
	\end{itemize}
	En cualquiera de estos dos casos, se sigue que $X[k{+}3]\,{\neq}\,X[k]$.

\item Supongamos que $X[k{+}1]\,{<}\, X[k]$. El razonamiento aquí es similar al caso anterior, para concluir que $X[k{+}3]\,{\neq}\,X[k]$.
\end{itemize}
En cualqueir caso, $X[k{+}3]\,{\neq}\,X[k]$, y como $k$ es arbitrario, entonces concluimos que si $X[k]\,{\neq}\, X[k{+}1]\,{\neq}\, X[k{+}2]$, entonces $X[k{+}3]\,{\neq}\,X[k]$, para todo $k\,{\in}\,\mathbb{N}$.
\end{proof}

\begin{lema}\label{lema3}
Si $X[k]\,{<}\,X[l]\,{<}\,X[k{+}1]$ y $X[l]\,{<}\,X[l{+}1]$, entonces $X[l{+}1]\,{<}\,X[k{+}1]$.
\end{lema}

\begin{proof}
Supongamos que:
\begin{itemize}
\item $X[k]\,{<}\,X[l]$
\item $X[l]\,{<}\,X[l{+}1]$
\item $X[l]\,{<}\,X[k{+}1]$
\end{itemize}

\[
x_i[l{+}1]\leq x_i[k{+}1]
\]

\[
u_i<X[k]<X[l]
\]

\[
u_i<X[l]
\]

\[
x_i[l+1]=0
\]


\end{proof}

\begin{teor}
Si existe un $k\,{\in}\,\mathbb{N}$ tal que $X[k]=X[k{+}1]$, entonces existe un $p\in[0,1]$ tal que
\[p*I=|\{i\,{\in}\,\mathcal{I}:p\leq u_i\}|\] 
\end{teor}
\begin{proof}

Supongamos que existe un $k\,{\in}\,\mathbb{N}$ tal que $X[k]=X[k{+}1]$ y sea $p=\frac{X[k]}{I}$. Observe que $p\in[0,1]$ y que $p*I=\frac{X[k]}{I}*I=X[k]=X[k{+}1]$. Además, observe que:
\begin{align*}
p*I&=X[k{+}1]\\
&=|\{i\,{\in}\,\mathcal{I}:\frac{X[k]}{I}\leq u_i\}| &\qquad\mbox{(por la observación 3)}\\
&=|\{i\,{\in}\,\mathcal{I}:p\leq u_i\}| &\qquad\mbox{(por definición de $p$)}\\
\end{align*}
\end{proof}

\begin{obs}
Si un sistema no tiene un estado estable, entonces para ningún $p\in[0,1]$ se cumple la igualdad.
\end{obs}

\begin{obs}
Para que se cumpla la igualdad, $p*I$ debe ser entero, luego consideramos los casos $p=\frac{j}{I}$, para $j\in\{1,\ldots,I\}$.
\end{obs}

\end{document}

