\documentclass{article}
\usepackage[utf8]{inputenc}

\begin{document}

Sistema de control de temperatura de un panal de abejas. Sistema de recolección de alimento de un hormiguero. Interacción entre las acciones de los individuos y un ``estado macro'' (temperatura, rastros de feromonas en el entorno, etc.). 

Modelo para estudiar la interacción elemental entre los ``estados micro'', es decir, las acciones de los individuos, y un estado macro (Elementary Micro Macro Interaction, EMMI). 

\

\begin{itemize}
\item[Agentes:] Sea $x$ un agente.

	\begin{enumerate}
	\item Acción $a_x\in\{0, 1\}$
	\item Umbral $u_x\in[0,1]$
	\item Regla $r_x\in\{(P>u_x)\leftrightarrow a_x=0, (P> u_x)\leftrightarrow a_x=1\}$
	\end{enumerate}

\item[Estados macro:] $P$, proporción $a_x=1$ sobre el total de agentes

\item[Dinámica:] Acción en $a_x(t+1)$ de cada agente se obtiene así: se calcula el estado macro  en el estado $t$ dependiendo de las acciones $a_x(t)$ de todos los agentes. Cada agente $x$ obtiene su acción $a_x(t+1)$ con base en su regla $r_x$.

\item[Población inicial:] Número de agentes; decisión inicial de agentes; regla inicial de agentes

\item[Aprendizaje:] 

\end{itemize}


\end{document}