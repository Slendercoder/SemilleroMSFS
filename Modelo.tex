\documentclass{article}
\usepackage[utf8]{inputenc}
\usepackage{amssymb, amsmath,bm,amsthm}

\newtheorem{lema}{Lema}

\begin{document}

Motivación: 
\begin{itemize}
\item Sistema de control de temperatura de un panal de abejas. Sistema de recolección de alimento de un hormiguero. Interacción entre las acciones de los individuos y un ``estado macro'' (temperatura, rastros de feromonas en el entorno, etc.).
\item Desarrollar un sistema formal sencillo para  estudiar la interacción elemental entre el estado micro (p.ej., la colección de los estados de las abejas en un instante determinado), y un estado macro (p.ej., la temperatura del panal) (Elementary Micro Macro Interaction, EMMI). 
\item Comparación con los autómatas celulares.
\end{itemize}

\

Definición de un EMMI:
\begin{itemize}
\item Sea $\mathcal{I}=\{1,\ldots,I\}$ un conjunto de agentes. Para cada $i\,{\in}\,I$ se define:
	\begin{enumerate}
	\item Umbral $u_i\in[0,1]$.
	\item Estado $x_i[k]\in\{0, 1\}$, para $k\in\mathbb{N}$.
	\item Regla  $x_i[k+1]=\begin{cases} 1, & \mbox{ si }X[k]\leq u_i\\ 0, & \mbox{ en otro caso }\end{cases}$
	\end{enumerate}

\item Estado macro: $X[k]=\sum_{i\in I} x_i[k]$, para $k\in\mathbb{N}$.
\end{itemize}

\begin{lema}
No puede haber dos transiciones consecutivas a estados de mayor número de individuos. Es decir, si $X[k]\,{\leq}\,X[k{+}1]$, entonces $X[k{+}2]\,{\leq}\, X[k{+}1]$.
\end{lema}
\begin{proof}
Supongamos que $X[k]\,{\leq}\,X[k{+}1]$. Vamos a demostrar primero que $x_i[k{+}2]\,{\leq}\, x_i[k{+}1]$ para todo $i$. Tenemos sólo dos casos:

\begin{itemize}
\item Supongamos que $x_i[k{+}1]=0$. Luego, por la definición de $x_i[k{+}1]$ y la hipótesis se tiene que $u_i<X[k]\leq X[k{+}1]$ y, en consecuencia, $x_i[k{+}2]=0$. Entonces $x_i[k{+}2]\leq x_i[k{+}1]$.
\item Supongamos que $x_i[k{+}1]=1$. Luego, como $x_i[k{+}2]\in\{0,1\}$, $x_i[k{+}2]\leq x_i[k{+}1]$.
\end{itemize}

Entonces $\sum_i x_i[k{+}2]\leq \sum_i x_i[k{+}1]$. Por lo tanto, por la definición de $X[k]$ se tiene que $X[k{+}2]\leq X[k{+}1]$.
\end{proof}

\end{document}

